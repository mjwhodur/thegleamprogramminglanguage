\chapter{Wstęp}
\paragraph{} Język programowania Gleam jest językiem stosunkowo nowym, który wersję stabilną (1.0.0) osiągnął 4 marca 2024 roku. Został stworzony przez Louisa Pilfolda. Na stronie internetowej  \texttt{https://gleam.run} można znaleźć najnowsze informacje na temat języka programowania.
\paragraph{} Jest to nowoczesny język kompilowany, a programy stworzone w nim mogą być uruchamiane w maszynie wirtualnej Erlanga - BEAM, a także w środowisku uruchomieniowym JavaScript.
\paragraph{} Celem tej publikacji jest opisanie języka i jego możliwości, a także wprowadzenie nowych użytkowników, których celem jest rozpoczęcie przygody z programowaniem i uruchamianiem kodu na maszynie wirtualnej Erlanga. 
\paragraph{} Podręcznik traktuje o języku, jako takim, a także o środowisku uruchomieniowym, którego zachowanie jest dosyć specyficzne. Istnieje wiele różnic pomiędzy klasycznymi programami (tzw. natywnymi), a tymi, które są uruchamiane na \textit{BEAM}. Sama maszyna wirtualna Erlanga posiada część cech systemu operacyjnego, to jest potrafi na przykład rozpoznaje aplikacje, umożliwia ich uruchamianie, ponowne uruchamianie i zatrzymywanie. Posiada również wiele specyficznych dla siebie cech, jak na przykład \textit{procesy}. Cechy te omówione zostaną w dalszych rozdziałach traktujących o programowaniu dla platformy \textit{BEAM}. Posiada również funkcje umożliwiające obliczenia rozproszone, na przykład tworzenie klastrów, a dzięki temu, uruchamianie zadań na innych węzłach i równoważenie obciążenia. Posiada również pewne cechy orkiestratorów. 
\paragraph{} Ta książka jest otwartoźródłowa. Jest licencjonowana na zasadzie Creative Commons - BY-SA. Derywaty tej publikacji muszą być publikowane na tej samej licencji, a prace pochodne muszą zawierać informacje o oryginalnych autorach.